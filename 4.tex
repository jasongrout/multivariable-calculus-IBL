

== Problem ==
There are two parts to this problem.
# Consider the point \(P\) with Cartesian (rectangular) coordinates \((2,1)\).  Find the distance \(r\) from \(P\) to the origin. Consider the ray \(\vec {OP}\) from the origin through \(P\). Find an angle between \(\vec{OP}\) and the \(x\)-axis. 
# Suppose that a point \(Q=(a,b)\) is 6 units from the origin, and the angle the ray \(\vec {OP}\) makes with the \(x\)-axis is \(\pi/4\) radians.  Find the Cartesian (rectangular) coordinates \((a,b)\) of \(Q\).
 
== Solution ==


== Problem ==
The following points are given using their polar coordinates.  Plot the points in the Cartesian plane, and give the Cartesian (rectangular) coordinates of each point. The points are
\[
(1,\pi), 
\displaystyle \left( 3,\frac{5\pi}{4}\right),
\displaystyle \left( -3,\frac{\pi}{4}\right),\text{ and }
\displaystyle \left( -2,-\frac{\pi}{6}\right).\]
== Solution ==


== Problem ==
Suppose that \(Q\) is a point in the plane with Cartesian coordinates \((x,y)\) and polar coordinates \((r,\theta)\).  Write formulas for \(x\) and \(y\) in terms of \(r\) and \(\theta\).  Then write a formula to find the distance \(r\) from \(Q\) to the origin (in terms of \(x\) and \(y\)) as well as a formula to find the angle \(\theta\) between the \(x\)-axis and a line connecting \(Q\) to the origin. [Hint: A picture of a triangle will help here.]
== Solution ==


== Problem ==
Consider the coordinate transformation \[\vec T(r,\theta) = (r\cos\theta,r\sin\theta).\] 
# Let \(r=3\) and then graph \(\vec T(3,\theta)=(3\cos\theta,3\sin\theta)\) for \(\theta\in[0,2\pi]\).
# Let \(\theta=\frac{\pi}{4}\) and then, on the same axes as above, add the graph of 
\(\vec T\left(r,\frac{\pi}{4}\right)=\left(r\frac{\sqrt 2}{2},r \frac{\sqrt 2}{2}\right)\) for \(r\in[0,5]\).
# To the same axes as above, add the graphs of 
\(\vec T(1,\theta), \vec T(2,\theta), \vec T(4,\theta)\)  for \(\theta\in[0,2\pi]\) and 
\(\vec T(r,0), \vec T(r,\pi/2), \vec T(r,3\pi/4), \vec T(r,\pi)\) for \(r\in[0,5]\). 
== Solution ==


== Problem == 
In the plane, graph the curve \(y=\sin x\) for \(x\in[0,2\pi]\) and the curve \(r=\sin\theta\) for \(\theta\in[0,2\pi]\) (just make an \(r,\theta\) table). 
== Solution ==

== Problem ==
Each of the following equations is written in the Cartesian (rectangular) coordinate system.  Convert each to an equation in polar coordinates, and then solve for \(r\) so that the equation is in the form \(r=f(\theta)\).
# \(x^2+y^2=7\)
# \(2x+3y=5\)
# \(x^2=y\)
== Solution ==

== Problem ==
Each of the following equations is written in the polar coordinate system.  Convert each to an equation in the Cartesian coordinates.
# \(r=9\cos\theta\)
# \(\displaystyle r=\frac{4}{2\cos\theta+3\sin\theta}\)
# \(\theta = 3\pi/4\)
== Solution ==



== Problem ==
Graph the polar curve \(r=2+2\cos\theta\).
== Solution ==

== Problem ==
Graph the polar curve \(r=2\sin 3\theta\).
== Solution ==

== Problem ==
Graph the polar curve \(r=3\cos 2\theta\).
== Solution ==


== Problem ==
Find the points of intersection of \(r=3-3\cos\theta\) and \(r=3\cos\theta\). (If you don't graph the curves, you'll probably miss a few points of intersection.)
== Solution ==

== Problem ==
Find the points of intersection of \(r=2\cos 2\theta \) and \(r=\sqrt 3\). (If you don't graph the curves, you'll probably miss a few points of intersection.)
== Solution ==


== Problem ==
Consider the polar curve \(r=1+2\cos \theta\). (It wouldn't hurt to provide a quick sketch of the curve.)
# Compute both \(dx/d\theta\) and \(dy/d\theta\).
# Find the slope \(dy/dx\) of the curve at \(\theta=\pi/2\).
# Give both a vector equation of the tangent line, and a Cartesian equation of the tangent line at \(\theta=\pi/2\).
== Solution ==


== Problem ==
Recall that \(x=r\cos\theta\) and \(y=r\sin\theta\). Suppose that \(r=f(\theta)\) for \(\theta\in[\alpha,\beta]\) is a continuous function, and that \(f'\) is continuous.  
Show that the arc length formula can be simplified to 
\[
s=\int_{\alpha}^{\beta}\sqrt{\left(\frac{dx}{d\theta}\right)^2+\left(\frac{dy}{d\theta}\right)^2} 
= \int_{\alpha}^{\beta}\sqrt{r^2+\left(\frac{dr}{d\theta}\right)^2} .\]
[Hint: the product rule and Pythagorean identity will help.]
== Solution ==

== Problem ==
Set up (do not evaluate) an integral formula to compute the length of 
# the rose \(r=2\cos 3\theta\), and
# the rose \(r=3\sin 2\theta\).
== Solution ==


== Problem ==
In this problem, you will develop a formula for finding area inside a polar curve.
#Consider a circle of radius \(r\). The area inside the circle is \(\pi r^2\). This is the area inside when you traverse around the circle for a full \(2\pi\) radians.  Fill in the following table by finding the pattern that connects angle traversed to area inside.
\begin{center}
\begin{tabular}{c|c}
Angle traversed& Area inside\\ \hline
\(2\pi\) & \(A=\pi r^2\)\\
\(\pi\) & \\
\(\pi/2\) & \\
\(\pi/4\) & \\
\(d\theta\) & \(dA=\quad\quad\)
\end{tabular}
\end{center}
# Explain why the area inside a polar curve \(r=f(\theta)\) for \(\alpha\leq \theta\leq \beta\) is \[A = \int dA = \int_\alpha^\beta \frac{1}{2}r^2d\theta.\] What must be true about the curve \(r=f(\theta)\) for this formula to be valid?

== Solution ==

== Problem ==
Find the area inside of the polar curve \(r=\sin\theta\). [Hint: Construct a graph to determine the appropriate bounds for the integral. When you integrate, you'll need to use the half angle identity.] 
== Solution ==


== Problem ==
Set up (do not evaluate) an integral to compute the area 
# inside the cardioid \(r=2+2\sin\theta\), and 
# inside the circle \(r=3\cos\theta\).
== Solution ==

== Problem ==
Set up (do not evaluate) an integral formula to compute the area that lies inside both \(r=2-2\cos\theta\) and \(r=\cos\theta\). Sketch both curves. 
== Solution ==

== Problem ==
Consider the coordinate transformation \(T(a,\omega)=(a\cos\omega,a^2\sin \omega)\).
#Let \(a=3\) and then graph the curve \(\vec T(3,\omega)=(3\cos\omega,9\sin\omega)\) for \(\omega\in[0,2\pi]\).
#Let \(\theta=\frac{\pi}{4}\) and then, on the same axes as above, add the graph of \(\vec T\left(a,\frac{\pi}{4}\right)=\left(a\frac{\sqrt 2}{2},a^2 \frac{\sqrt 2}{2}\right)\) for \(a\in[0,4]\).
#To the same axes as above, add the graphs of \(\vec T(1,\omega), \vec T(2,\omega), \vec T(4,\omega)\)  for \(\omega\in[0,2\pi]\) and \(\vec T(a,0), \vec T(a,\pi/2), \vec T(a,-\pi/6)\) for \(a\in[0,4]\). 
[Hint: when you're done, you should have a bunch of parabolas and ellipses.]
== Solution ==

== Problem ==
Let \(P=(x,y,z)\) be a point in space. This point lies on a cylinder of radius \(r\), where the cylinder has the \(z\) axis as its axis of symmetry.  The height of the point is \(z\) units up from the \(xy\) plane. The point casts a shadow in the \(xy\) plane at \(Q=(x,y,0)\).  The angle between the ray \(\vec{OQ}\) and the \(x\)-axis is \(\theta\). Construct a graph in 3D of this information, and use it to develop the equations for cylindrical coordinates given above.
== Solution ==

== Problem == 
Let
\(P=(x,y,z)\) be a point in space. This point lies on a sphere of
radius \(\rho\) (``rho''), where the sphere's center is at the origin
\(O=(0,0,0)\). The point casts a shadow in the \(xy\) plane at
\(Q=(x,y,0)\).  The angle between the ray \(\vec{OQ}\) and the \(x\)-axis
is \(\theta\), and is called the azimuth angle. The angle between
the ray \(\vec{OP}\) and the \(z\) axis is \(\phi\) (``phi''), and is
called the inclination angle, polar angle, or zenith angle.  Construct
a graph in 3D of this information, and use it to develop the
equations for spherical coordinates given above.
== Solution ==

