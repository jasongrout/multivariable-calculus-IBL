{\huge
Don't print this chapter yet.  Please wait until I remove this comment, as I will be rewriting the chapter over the next few days.
}

\noindent 
This unit covers the following ideas. In preparation for the quiz and exam, make sure you have a lesson plan containing examples that explain and illustrate the following concepts.  
\begin{enumerate}
 \item Develop formulas for the velocity and position of a projectile, if we neglect air resistance and consider only acceleration due to gravity. Show how to find the range, maximum height, and flight time of the projectile.
 \item Develop the $TNB$ frame for describing motion. Make sure you can explain why $\vec T$, $\vec N$, and $\vec B$ are all orthogonal unit vectors, and be able to perform the computations to find these three vectors.
 \item Explain the concepts of curvature $\kappa$, radius of curvature $\rho$, center of curvature, and torsion $\tau$. Make sure you can describe geometrically what theses quantities mean.
 \item Find the tangential and normal components of acceleration. Show how to obtain the formulas $a_T=\frac{d}{dt}|\vec v|$ and $a_N=\kappa |\vec v|^2=\frac{|\vec v|^2}{\rho}$, and explain what these equations physically imply.
\end{enumerate}
You'll have a chance to teach your examples to your peers prior to the exam.

I have created a YouTube playlist to go along with this section. There are 11 videos, each 4-6 minutes long.
\begin{itemize}
 \item \href{http://www.youtube.com/playlist?list=PL30EE81142B1ED1F0&feature=plcp}{YouTube playlist for 07 - Motion and The TNB Frame}.
 \item \href{http://db.tt/FmEGk9p5}{PDF copy of the finished product} (so you can follow along on paper).
\end{itemize}
Table \ref{motion table} summarizes most of the concepts we'll discuss. The goal of this chapter is to explain how the vectors in this table are related. 

\begin{table}
\begin{center}
\begin{tabular}{|c|c|c|}
\hline
Quantity & Symbol & Formula\\\hline\hline
Position (``r''adial vector) & $\vec r$ & $\vec r(t) = (x(t),y(t),z(t))$\\\hline
Velocity  & $\vec v$ & $\ds \vec v(t) = \frac{d\vec r}{dt}$\\\hline
Speed  & $v$ & $\ds v(t) = |\vec v(t)|$\\\hline
Acceleration  & $\ds \vec a$ & $\ds \vec a(t) = \frac{d \vec v}{dt}= \frac{d^2\vec r}{dt^2}= \frac{d}{dt}\frac{d\vec r}{dt}$\\\hline
Unit Tangent Vector & $\vec T$ & $\ds\frac{d\vec r}{ds} = \frac{d\vec r/dt}{ds/dt} = \frac{\vec r^\prime(t)}{|\vec r^\prime(t)|}$\\\hline
Curvature Vector & $\vec \kappa $& $\ds\frac{d\vec T}{ds} =\frac{d\vec T/dt}{ds/dt} = \frac{d\vec T/dt}{|\vec v|} = \frac{\vec T^\prime(t)}{|\vec r^\prime(t)|} $\\\hline
Curvature (a scalar)& $ \kappa $&$\ds \left|\frac{d\vec T}{ds}\right| =\left|\frac{d\vec T/dt}{ds/dt}\right| = \frac{\left|d\vec T/dt\right|}{|\vec v|}= \frac{|\vec T^\prime(t)|}{|\vec r^\prime(t)|}  $ \\\hline
Curvature of $y=f(x)$& $ \kappa(x) $&$\ds \kappa(x) = \frac{|f''(x)|}{(1+(f')^2)^{3/2}}.  $ \\\hline
Principal unit normal vector & $ \vec N$& $\ds \frac{d\vec T/dt}{|d\vec T/dt|} =  \frac{\vec T^\prime(t)}{|\vec T^\prime(t)|}=\frac{1}{\kappa}\frac{d\vec T}{ds} = \frac{1}{\kappa |\vec v|}\frac{d\vec T}{dt}$\\\hline
Binormal vector & $ \vec B$& $ \vec T\times\vec N$\\\hline
Radius of curvature & $ \rho$ & $1/\kappa$\\\hline
Center of curvature &  & $\vec r(t)+\rho(t)\vec N(t)$ \\\hline
Torsion & $ \tau $ & $\ds \pm\left|\frac{d\vec B}{ds}\right|$ (pick the sign) or $\ds-\frac{d\vec B}{ds}\cdot \vec N $\\\hline
Tangential Component of acceleration & $ a_T$ & $\ds \vec a \cdot \vec T = \frac{d}{dt}|\vec v|$\\\hline
Normal Component of acceleration & $ a_N$ & $\ds \vec a \cdot \vec N = \kappa \left(\frac{ds}{dt}\right)^2 = \kappa |\vec v|^2$\\\hline
Acceleration (sum the components)& $ \vec a$ & 
$\vec a 
= a_T\vec T+a_N\vec N 
= \left(\frac{d}{dt}|\vec v|\right) \vec T 
 +\left(\kappa |\vec v|^2\right) \vec N  $\\\hline


\end{tabular}
\caption{This table summarizes the key ideas in this unit. Most of our work in this chapter will be to explain the connections between these variables.\label{motion table}} 
\end{center}
\end{table}


\section{Projectile Motion}

Have you ever dropped a rock from the top of a waterfall, or skipped a rock across a lake. This section explores some simple connections between position, velocity, and acceleration. If we wanted to send a rocket to space, or shoot a missile across an ocean, the same principles will apply. If we know how much thrust a rocket provides (the acceleration), can we determine the velocity of our rocket at any time along its path?  Could we predict the flight path of the rocket? To make a good flight plan, we'd need to know how to determine position and velocity from acceleration.  That's the content of this section.  

\begin{review*}
 If $y'(t) = 3t^2+12e^{2t}$ (the velocity) and $y(0)=2$ (initial height), then what is $y(t)$? See footnote \footnote{Integrate to get $y(t) = t^3+6e^{2t}+C$. Since $y(0)=2$, we know $2=0+6(1)+C$, which gives $C=-4$. So the height is $y(t) = t^3+6e^{2t}-4$. } for an answer. 
\end{review*}


To solve the next problem, we need to remember that acceleration is the derivative of velocity, and that velocity is the derivative of position.  These facts hold true for vector-valued functions as well.


\begin{problem}
Consider a rocket in space (so we can neglect air resistance and gravity). The rocket's boosters apply an acceleration $\vec a(t) = (2t,-8)$ m/s$^2$. The rocket's initial velocity is $\vec v(0) = (4,5)$ m/s.  The initial position is $\vec r(0) = (1,)$ m. Use this information to determine the position of the object after 2 seconds, and after 3 seconds. 

[Hint: Integrate each component to get velocity, then repeat to get position. Don't forget the 4 arbitrary constants you get from integration. Use the initial velocity and initial position to determine these constants.]
\end{problem}

Suppose we fire a projectile (like a pumpkin) from a cannon. The projectile leaves the cannon with an initial speed $v_0$, at an angle of $\alpha$ above the $x$-axis. All the motion in this problem occurs with a plane, and we'll use $x$ and $y$ to represent motion in that plane. Our goal is to find the velocity $\vec v(t)$ and position $\vec r(t)$  of the projectile at any time $t$. 

We need some assumptions prior to solving. 
\begin{itemize}
 \item Assume the only force acting on the object is the force due to gravity. We will neglect air resistance. 
 \item The force due to gravity is the mass of the projectile multiplied by the acceleration of gravity. The mass of the object will not be important in our work here, though in future classes you may study how mass affects energy computations. 
 \item The projectile is shot over a small enough range that we can assume gravity only pulls the object straight down.
 \item Most branches of science use the letter $g$ to represent the magnitude of the vertical component of acceleration, so we can write the acceleration of the projectile as 
$$\vec a(t) = (0,-g) \quad \quad \text{or}\quad quad \vec a(t)= 0{\bf i}-g{\bf j}.$$ 
 \item Our text uses the approximations $g\approx 9.8$ m/s$^2$ or $g\approx32$ ft/s$^2$. 
\end{itemize}


You've probably heard before that when you throw a baseball to a friend, the path of the baseball is parabolic. The next problem proves this. If you feel shaky on getting a Cartesian equation from a parametrization, please tackle this review problem, otherwise, jump straight to the problem.
\begin{review*}
 The function $\vec r(t) = (2t+3, 4t^2+7t+5)$ is a parametrization of a plane curve.  Give a Cartesian equation of the curve. 
 See \footnote{Since $t=\dfrac{x-3}{2}$, a Cartesian equation is $y = 4\left(\dfrac{x-3}{2}\right)^2+7\left(\dfrac{x-3}{2}\right)+5$. } for an answer.
\end{review*}


\begin{problem}\marginpar{Watch a \href{http://www.youtube.com/watch?v=dW0bm7cLB8E&list=PL30EE81142B1ED1F0&index=1&feature=plpp_video}{YouTube video}.}%
\marginparbmw{You can practice finding position from velocity and acceleration with problems 13.2: 11-18, and especially 13.2: 29.}
Suppose a projectile is fired from the point $(x_0,y_0)$ with an initial velocity $\vec v(0)=(v_{x_0},v_{y_0})$, and that gravity is the only force acting on the object. This means the acceleration on the object is $\vec a(t) = (0,-g)$.
\begin{enumerate}
 \item Show that the velocity at any time $t$ is $\vec v(t) = (c_1,-gt+c_2)$. What are $c_1$ and $c_2$? Explain
 \item Show that the position at any time $t$ is $\vec r(t) = (v_{x_0}t+c_3, -\frac{1}{2}gt^2+v_{y_0}t+c_4)$. What are $c_3$ and $c_4$? 
% \item Give parametric equations $x=x(t)$ and $y=y(t)$ that give the horizontal and vertical position of the projectile at time $t$. 
 \item Eliminate the parameter $t$ to give a Cartesian equation of the projectile's path. This will prove that the path of the particle is parabolic.
\end{enumerate}
\end{problem}

If a projectile starts at $(x_0,y_0)$, we can move the origin to this point. As long as we are not trying to gauge the location of two projectiles simultaneously, we could always make the origin $(0,0)$ our starting point.   
We make the following definitions for a projectile that starts at $(0,0)$ and hits the ground at $(R,0)$.
\begin{itemize}
 \item The range is the horizontal distance $R$ traveled by the projectile.  
 \item The flight time is how long the projectile is in the air. It is the time $t$ at which $\vec r(t)=(R,0)$.
 \item The maximum height is the largest $y$ value obtained by the projectile. 
\end{itemize}


\begin{problem}\marginpar{Watch a \href{http://www.youtube.com/watch?v=a6PHAvynNWM&list=PL30EE81142B1ED1F0&index=2&feature=plpp_video}{YouTube video}.}%
 Answer the following questions. Assume we fire a projectile from the origin, which means the acceleration, velocity, and position are 
$$ \vec a(t) = (0,-g),\quad 
\vec v(t) = (v_{x_0}, -gt+ v_{y_0}),\quad
\vec r(t) = (v_{x_0}t, -\frac12 gt^2+ v_{y_0}t)
.$$
\begin{enumerate}
 %\item What should the velocity vector equal when the object has reached the maximum height?
 \item What's the time to max height?  What's the flight time? 
 \item Show why the maximum height is $\ds y_{\max}=\frac{v_{y_0}^2}{2g}$ and the range is $\ds R=\frac{2v_{x_0}v_{y_0}}{g}$.
 \item If the initial speed is $v_0$, with a firing angle of $\alpha$ above the horizontal, rewrite $v_{x_0}$ and $v_{y_0}$ in terms of $v_0$ and $\alpha$, and then state the range in terms of $v_0$ and $\alpha$. 
\end{enumerate}
\end{problem}

The next problem comes from your text. (See section 13.2.)  Try it without reading the text.  It's a fun application of the ideas above.
\begin{problem}\marginpar{This problem was created around the opening ceremony of the Barcelona Spain Olympics.  Antonio Rebollo was the archer, but he didn't try to hit the flame at the peak of the flight. You can \href{http://www.youtube.com/watch?v=b5gZeT4TVds}{watch a YouTube video} of the opening ceremony by following the link.}%
\marginparbmw{See 13.2: 19-28 for more practice.}  
 An archer stands at ground level and shoots an arrow at an object which is 90 feet away in the horizontal direction and 74 ft above ground. The arrow leaves the bow at about 6 ft above ground level (not the origin). 
 The archer wants the arrow to hit the target at the peak of its parabolic path. 
 For the purposes of this problem, Let $g = 32 \text{ft}/\text{s}^2$. 
 What initial speed $v_0$ and firing angle $\alpha$ are needed to achieve this result? 
 [Hint: This is much easier to solve if you first find $v_{x_0}$ and $v_{y_0}$, the horizontal and vertical components of the velocity. You may want to move the origin as well, so that you can use the formulas from above.]
\end{problem}

\section{Arc Length and the Unit Tangent Vector}
Now that we've developed a way to predict position and velocity from acceleration, let's look more in depth at the actual path taken by a projectile. We'll need to be able to compute the actual distance an object travels (not the displacement, but the distance).  This requires that we study arc length.  We did this in chapter's 3 and 4 for curves in the plane.

\begin{review*}
 A horse runs once around an elliptical track, which is parametrized by $\vec r(t) = (3\cos t,4\sin t)$.  Set up, do not solve, an integral formula that tells us the distance the horse traveled. What's the displacement? See 
\footnote{The velocity is $\vec v(t) = (3\sin t, -4\cos t)$. The speed is $v(t) = \sqrt{9\sin^2t+16\cos^2t}$. The distance traveled is the arc length $\ds s=\int_0^{2\pi} \left(\sqrt{9\sin^2t+16\cos^2t}\right)dt$. Since the horse's initial and final position are equal, the displacement is zero. Arc length does not equal displacement. }
for an answer.
\end{review*}
 

Let's now develop a formula for the arc length of a space curve, a curve in 3D. We can always parameterize a space curve with $\vec r(t) = (x,y,z)$ (one input, 3 outputs).

\begin{problem}\marginpar{Watch a \href{http://www.youtube.com/watch?v=jZpAU2T6iI4&list=PL30EE81142B1ED1F0&index=3&feature=plpp_video}{YouTube video}.}
A space ship travels through the galaxy. Let $\vec r(t) = (x,y,z)$ 
\marginpar{Technically, we should write $\vec r(t) = (x(t),y(t),z(t))$. However, we already know that $x$, $y$, and $z$ depend on $t$, hence we'll just leave the dependence on $t$ off.}%
be the position of the space ship at time $t$, with the earth at the origin $(0,0,0)$. 
\begin{itemize}
 \item What are the velocity and speed of the space ship at time $t$? You answers should involve some derivatives (such as $\frac{dx}{dt}$).
 \item If the space ship travels for a really small time $dt$, then the speed is about constant. Since distance is speed times time, about how much distance (we'll call it $ds$) will the space ship travel in this short amount of time?
 \item As the ship travels from time $t=a$ to time $t=b$, explain why the distance traveled (the arc length of the path followed) is $$s=\int_a^b |\vec r '(t)|\ dt = \int_a^b \sqrt{\left(\frac{dx}{dt}\right)^2+\left(\frac{dy}{dt}\right)^2+\left(\frac{dz}{dt}\right)^2}\ dt .$$
\end{itemize}

\end{problem}

In all our work that follows, we want to consider space curves that have nice smooth paths.  What does this mean?  We want to be able to compute tangent vectors at any point, so we will require that a parametrization $\vec r$ be differentiable.  However, this isn't enough.  
\begin{problem}
 We've encountered the polar curve $r = 1-\sin\theta$ before (we called it a cardioid, and it looked like heart).  Recall that we can switch from polar to Cartesian using the coordinate transformation $x=r\cos\theta$ and $y=r\sin\theta$. 
\begin{enumerate}
 \item Draw the curve.
 \item A parametrization of this curve is $\vec r(\theta) = ((1-\sin\theta)\cos\theta, ?)$.  This parametrization is completely differentiable.  Find $\dfrac{d\vec r}{d\theta}$.
 \item You should notice a sharp cusp in the graph. At what $\theta$ does this cusp occur?  What is the value of the derivative $\dfrac{d\vec r}{d\theta}$ at this value of $\theta$. 
\end{enumerate}
\end{problem}

We'd like to avoid paths that contain a cusp, because at a cusp the direction of motion changes rather abruptly. This can happen physically, but it requires the speed of an object to reach zero, the object stops moving, and then the path changes direction. The fact that the speed reaches zero will mean we can't divide by it in our work that follows.  To avoid this, we make a definition that requires the path is differentiable, and the velocity is never zero.

\begin{definition}[Smooth Curves]
 Let $\vec r(t)=(x,y,z)$ be a parametrization of a space curve $C$. We say that $\vec r$ is smooth if $\vec r$ is differentiable, and the derivative is never the zero vector. If $\vec r$ is a smooth parameterization, then we call $C$ a smooth curve. 
\end{definition}


\begin{problem}\marginpar{Watch a \href{http://www.youtube.com/watch?v=m25oxYTfXfU&list=PL30EE81142B1ED1F0&index=4&feature=plpp_video}{YouTube Video}.  }%
\marginparbmw{See 13.3: 1-10 for more practice.}%
 Consider the helical space curve $C$ with parameterization $\vec r(t)=(\cos t, \sin t, t)$. 
\begin{enumerate}
 \item Is C a smooth curve?  
 \item Find the length of this space curve for $t\in[0,2\pi]$. Compute any integrals. 
 \item Now find the length of the space curve from $t=0$ to time $t=t$. 
 \item Give a vector of length 1 that is tangent to the curve at $t=2\pi$. 
\end{enumerate}
\end{problem}

In the previous problem, you developed two big ideas.  You showed how to obtain a unit tangent vector to a curve. You also developed a formula for the length of a curve from time $t=0$ to any time $t=t$.  This gives us a function $s(t)$ that tells us how far we have traveled after $t$ seconds. We can now predict distance traveled from time. Predicting the future is powerful. Before moving on, let's examine the derivative of $s(t)$, because it's a quantity we already know.

\begin{review*}
 Compute $\ds \int_0^t 3x^2 dx$ and $\ds \int_0^t 3p^2 dp$ and $\ds \int_0^t 3\tau^2 d\tau$. 
 Does it matter what you call the variable inside the integral? 
 Then compute $\ds\frac{d}{dt} \int_0^t 3\tau^2 d\tau$. See 
\footnote{
The first integral is $x^3|_0^t = t^3$. The other two are the same. You can change the variable inside the integral whenever you want.  For this reason, some people call it a dummy variable. 
The last part is $\ds\frac{d}{dt} \int_0^t 3\tau^2 d\tau = \frac{d}{dt} t^3 = 3t^2$, 
we just replaced $\tau$ with $t$ in $3\tau^2$.}
for an answer. 
\end{review*}


\begin{problem}\label{fundamental theorem of calculus as it applies to arc length parameter}\marginpar{You can remember $\ds\frac{ds}{dt} = \left|\frac {d\vec r}{dt} \right|$ as follows. We use the differential $ds$ to represents a change in distance, and $dt$ represents a change in time. So the speed of an object is the change in distance $ds$ over the change in time $dt$. }%
 Let $\vec r(t)=(x,y,z)$ be a parametrization of a smooth space curve. Let $\ds s(t)=\int_0^t \left|\frac {d\vec r}{d\tau} \right|\ d\tau$.  Explain why $\ds\frac{ds}{dt} = \left|\frac {d\vec r}{dt} \right|$, the speed. Then explain why $s(t)$ is an increasing function.

 [Hint: Look up the fundamental theorem of calculus. To answer why is $s$ increasing, what does ``smooth'' mean?] 
\end{problem}

We'll call $\ds s(t)=\int_0^t \left|\frac {d\vec r}{d\tau}\right|\ d\tau$ the arc length parameter.  It tells us how far we've have traveled after $t$ seconds. We can now predict distance traveled from time elapsed. Because $s(t)$ is an increasing function, we can also invert this process and give time elapsed from distance traveled. This means we could compute derivatives with respect to $s$ instead of $t$.  
When we take a derivative with respect to $s$, we ask how much a curve changes if we increase length by 1 unit, instead of increasing time by 1 unit.  We'll write
$$\ds\frac{d\vec r}{ds} =\ds\frac{d\vec r/dt}{ds/dt} = \frac{d\vec r/dt}{|d\vec r/dt|} = \frac{\vec v}{v}.$$

\begin{problem}\marginparbmw{See 13.3: 11-14 for more practice.}
Consider again the helical space curve $\vec r(t)=(\cos t, \sin t, t)$.  We've shown that $s(t) = t\sqrt{2}$. 
\begin{enumerate}
 \item Solve for $t$ in terms of $s$ (so find the inverse of $s(t)$). If you've traveled 4 units of distance, how much time has elapsed.   
 \item Compute $D\vec r(t)$ and $Dt(s)$.  You should have a 2 by 1 matrix, and a 1 by 1 matrix. 
 \item Use the chain rule to compute the derivative of $\vec r(t(s))$.
 \item Compute the length $\left|\ds\frac{d\vec r}{ds}\right|$.
\end{enumerate}
\end{problem}

The previous problem motivates the following definition.

\begin{definition}[Unit Tangent Vector]\label{def unit tangent vector}
 Let $\vec r(t)$ be a parametrization of a smooth space curve. We define the unit tangent vector $\vec T(t)$ to be the derivative of $\vec r$ with respect to arc length, which means
$$\vec T = \ds\frac{d\vec r}{ds}=\ds\frac{d\vec r/dt}{ds/dt} = \frac{d\vec r/dt}{|d \vec r/dt|} = \frac{\vec v}{|\vec v|}.$$
This is exactly the same as unit vector in the same direction as the velocity.
\end{definition}

As we progress through this unit, one of our key goals is to learn the new notation.  We've got position $\vec r$,  velocity $\vec v$, speed $v$ or $ds/dt$, acceleration $\vec a$, the unit tangent vector $\vec T$, the derivative of position with respect to arc length $d\vec r/ds$.  The last two are the exact same since $\vec T = d\vec r/ds$. Did you also notice that $ds/dt$ and $v$ are both the speed?  We'll need to start realizing that the same quantity can be developed in many ways. 

\begin{problem}
\marginparbmw{See 13.3: 1-10 for more practice.}%
 Suppose an object moves along the space curve given by  $\vec r(t)=(a\cos t,a\sin t,b t)$. 
\begin{enumerate}
 \item Find the object's velocity and speed. What is $ds/dt$?
 \item Compute $\frac{d\vec r}{ds}$, the derivative of $\vec r$ with respect to arc length. [Hint: Divide the top and bottom by $dt$ and then compute $d\vec r/dt$ and $ds/dt$.]  
 \item State the unit tangent vector $\vec T(t)$.
\end{enumerate}
\end{problem}


\section{Some Examples}

Let's look at a few examples to help motivate the remainder of our discussion.
\begin{problem}
 Consider the curve $\vec r(t) = (3\cos t, 3\sin t)$.
\begin{enumerate}
 \item Draw the curve. Compute $\vec v$ and $\vec a$. At the point $t=\pi/2$ draw these two vectors. Are these two vectors orthogonal? 
 \item Compute $\vec T$ and $\dfrac{d\vec T}{dt}$. At the point $t=\pi$ draw these two vectors. Are these two vectors orthogonal? 
 \item Compute $ds/dt$ and then $\dfrac{d\vec T}{ds}$. How long is $\dfrac{d\vec T}{ds}$?  
\end{enumerate}
\end{problem}

\begin{problem}
 Consider the curve $\vec r(t) = (t, t^2)$. The computations here can get intense, so please use Sage (follow this link) to help you. If you'll use the computer, this problem will go really fast.  
\begin{enumerate}
 \item Draw the curve. Compute $\vec v$ and $\vec a$, and then at the point $t=1$ draw these two vectors. Are $v$ and $a$ orthogonal?
 \item Compute $ds/dt$ and $\vec T$. 
 \item Compute $\dfrac{d\vec T}{dt}$ and $\dfrac{d\vec T}{ds}$. Then at the point $t=\pi$ draw the vector $d\vec T/dt$, and at the point $t=3\pi/2$ draw the vector $d\vec T/ds$.
 \item What's the difference between $\dfrac{d\vec T}{dt}$ and $\dfrac{d\vec T}{ds}$?
\end{enumerate}
\end{problem}




\section{The TNB Frame}
The unit tangent vector $\vec T$ provides us with a unit vector in the direction of motion. We can obtain the direction of motion from the velocity. If we stay on a straight course, then our acceleration is in the same direction as our motion, and would only cause us to speed up or slow down. We'll call this tangential acceleration.

If we want to design a roller coaster, build an F15 fighter plane, send a satellite in orbit, or construct anything that doesn't move in a straight line, we need to understand the connection between turning off a straight path, and acceleration. There will still be tangential acceleration, but now we'll have a component that veers us off the straight path.  This is called normal acceleration.  It's the acceleration that acts orthogonal to the tangential acceleration.  We would like to write the acceleration as the sum of it's tangential and normal parts, namely 
$$\vec a = \vec a_{\text{tangential}} + \vec a_{\text{normal}}.$$
Then we can study the tangential and normal parts separately.

\end{document}
\note{Here is a great spot to get some intuition.  Grab a problem from the end.  Use the projectile motion problem.}
Start in 2D.  Develop the components to start with by having them work a real example.  Ask them for T and N.  Give them r, have them compute v, T, and a, then ask for av and an.  This is projections.
They can see how to get N by swapping the order and changing the sign. This could be stinking cool!  
They won't get curvatuve here, but that's OK.
They won't need anymore in 2D.  

Then go to 3D.  This is where the crucial fact occurs about vectors of constant length.
The swap rule no longer applies (right).
Show how to get N.
Get B.
Motivate K and tau
obtain k and tau.
give cool way to get k and tau.
connect back to acceleration.
scrap center of curvatuve problem?  No, they need that for engineering.  But maybe use a parabola instead of trig function, or a cubic. I like the parabola because of it's nice answer.


When you study dynamics (forces acting on moving objects), you'll find that knowing the tangent and normal directions are crucial. In our class, we only have time to develop equations for $\vec T$ and $\vec N$, as well as practice on a few examples. 

In order to find $\vec N$, we first need to develop a crucial fact.  This fact states that if a vector valued function has constant length, then the function is orthogonal to its derivative. Here's an example. 

\begin{problem}
 Consider  $\vec r_1(t)=(\cos t, \sin t, 0)$ and $\vec r_2(t)=(\cos t, \sin t, t)$. 
\begin{enumerate}
 \item Show that $\vec r_1$ and $\dfrac{d\vec r_1}{dt}$ are orthogonal. Is $|\vec r_1|$ constant?
 \item Show that $\vec r_2$ and $\dfrac{d\vec r_2}{dt}$ are not orthogonal. Is $|\vec r_2|$ constant?
 \item Is the length of $\dfrac{d\vec r_2}{dt}$ constant? Are $\dfrac{d\vec r_2}{dt}$ and $\dfrac{d^2\vec r_2}{dt^2}$ orthogonal? 
\end{enumerate}
\end{problem}

\begin{theorem}\label{vector valued functions of constant length}
 If a vector valued function $\vec r(t)$ has constant length, then the vector $\vec r$ and its derivative $\ds\frac{d\vec r}{dt}$ are orthogonal for all $t$. 
\end{theorem}

\begin{problem}[Proof of Theorem \ref{vector valued functions of constant length}]\marginpar{Watch a \href{http://www.youtube.com/watch?v=08Ygw_M-4yM&list=PL30EE81142B1ED1F0&index=6&feature=plpp_video}{YouTube Video}.  }%
 Prove the theorem above. Here are some hints [as an alternative to watching the YouTube video].
\begin{itemize}
 \item We know that $\vec r(t)$ has constant length. This means $|\vec r|=c$ for some constant $c$. 
 \item You need to get from a magnitude to the dot product. Look in your text for a way to relate magnitude to the dot product. See problem \ref{dot product facts}.
 \item After writing $|\vec r(t)|=c$ in terms of a dot product (squaring both sides may help), take the derivative of both sides. Apply the product rule to the dot product.
\end{itemize}
\end{problem}

The above fact is so crucial, that we'll repeat what it says.
\begin{quote}
If the vector $\vec v(t)$ has constant length, then the vector and its derivative $\frac{d\vec v}{dt}$ are orthogonal.
\end{quote}


\begin{problem}\label{T and N are orthogonal}\marginpar{Watch a \href{http://www.youtube.com/watch?v=aJttU3kS_p8&list=PL30EE81142B1ED1F0&index=7&feature=plpp_video}{YouTube Video}.  }%
 Let $\vec r$ be a smooth parametrization of a curve.  How long is the {\it unit} tangent vector $\vec T(t)$? Explain why $\vec T$ is orthogonal to $\dfrac{d\vec T}{dt}$. 
 Give a formula for computing a unit vector that is orthogonal to $\vec T(t)$. 
\end{problem}

Based on your answer above, we make the following definition of the principle unit normal vector.  The key idea is that this vector points in the direction of normal acceleration. 
\begin{definition}[Principle Unit Normal Vector]
 If $\vec r(t)$ is a parametrization of a space curve with unit tangent vector $\vec T(t)$, then we define the principle unit normal vector $\vec N(t)$ to be the vector
 $$\vec N(t) = \ds\frac{d\vec T/dt}{|d\vec T/dt|},$$
 provided of course that $|d\vec T/dt|\neq 0$. 
 From problem \ref{T and N are orthogonal} we know that $\vec T$ and $\vec N$ are orthogonal.
\end{definition}

\begin{definition}[Binormal Vector]
 If $\vec r$ is a parametrization of a smooth space curve with unit tangent vector $\vec T$ and principle unit normal vector $\vec N$, then we define the binormal vector $\vec B$ to be the cross product
$$\vec B = \vec T\times \vec N.$$
\end{definition}

We now have the entire $TNB$ frame.  This gives us a moving collection of unit vectors that act like an $xyz$ coordinate system.  Many of you will use this frame a ton in your dynamics course. The TNB frame shows up in physical chemistry as well. A key fact to remember is that all three vectors are unit vectors, and they are each orthogonal to the other.

\begin{problem}
Answer the following questions (this will review your knowledge of the dot and cross products).
\begin{enumerate}
 \item What is $\vec T\cdot \vec N$? Explain. Then explain why $\vec T\cdot \vec B=0$ and $\vec N\cdot \vec B=0$.
 \item Both $\vec T$ and $\vec N$ are unit vectors. Why is $\vec B$ is a unit vector? [Think about the connection between the cross product and area.] 
 \item We defined $\vec B=\vec T\times \vec N$. This means that $\vec N\times \vec T=-\vec B$.  Is $\vec B\times \vec T$ equal to $\vec N$ or $-\vec N$? Explain.
\end{enumerate}
\end{problem}

\begin{problem} \label{helix example of T N and B}
\marginparbmw{See 13.4: 9-16 and 13.5: 9-16 (the relevant parts) for more practice.}%
Consider the helix $\vec r(t) = (3\cos t,3\sin t, 4t)$.  Find the unit tangent vector $\vec T(t)$, principle unit normal vector $\vec N(t)$, and the binormal vector $\vec B(t)$. 
\end{problem}

We've been working with helices in all the problems up to now because the velocity vectors have constant speed.  Once the speed of the velocity vector is no longer constant, things get a lot messier. Ask me in class to show you what happens with the computations when you consider something like $r(t)=(t,t^2,t^3)$. Things get ugly really fast. Fortunately, when you're working with a curve that lies in a plane, there are some simplifications that occur.

\begin{problem}
\marginparbmw{See 13.4: 7-8 for more practice, and perhaps a hint.}%
 Suppose you have already computed the unit tangent vector for a curve in the plane and found at a specific time it equals $\vec T=(a,b)$, which could easily be rewritten as $\vec T = (a,b,0)$.   
\begin{enumerate}
 \item Find a nonzero vector that is orthogonal to $\vec T=(a,b)$. 
 \item If $\vec r(t) = (t,t^2)$, then we have $\frac{\vec r}{dt} = (1,2t)$ and $\vec T(t) = \frac{(1,2t)}{\sqrt{1+4t^2}}$. Without computing any more derivatives, what is the principle unit normal vector $\vec N(t)$? Draw a picture of the curve, and then at $t=1$ add to your picture the tangent and normal vectors. 
 \item  What is $\vec B(t)$? We'll answer this in class if you are not sure.
\end{enumerate}
\end{problem}

\begin{observation}
From the problem above, we learn the following fact.  If the tangent vector to a planar curve is $\vec T(t) = (a(t),b(t))$, then the principle unit normal vector is either $\vec N(t)=(-b(t),a(t))$ or $\vec N(t)=(b(t),-a(t))$.  You just reverse the components, and then negate one of them.  To determine which one to negate, draw a picture.
\end{observation}

\begin{problem}
Consider the curve $\vec r(t)=(t^2,t)$. Compute $\vec T(t)$ and $\vec N(t)$ (to get $\vec N(t)$, make sure you use the previous observation). Draw the curve and on your graph include these vectors at $t=1$. 
\end{problem}



\begin{problem}
\marginparbmw{See 13.4: 1-4 for more practice. Use the previous problems.}%
 Consider the curve $y=\sin x$, parametrized by $r(t)=(t,\sin t)$. Start by computing $\vec T(t)$.
\begin{enumerate}
 \item What are $\vec T(t)$, $\vec N(t)$, and $\vec B(t)$ at $t=\pi/2$?
 \item What are $\vec T(t)$, $\vec N(t)$, and $\vec B(t)$ at $t=\pi/4$?
 \item What are $\vec T(t)$, $\vec N(t)$, and $\vec B(t)$ at $t=-\pi/4$?
 \item What are $\vec T(t)$, $\vec N(t)$, and $\vec B(t)$ at $t=0$?
\end{enumerate}
\end{problem}


You've now developed the TNB frame for describing motion. Engineers will see this again when they study dynamics. Mathematicians who study differential geometry will use these ideas as well. Any time you want to analyze the forces acting on a moving object, the TNB frame may save the day. Chemists will encounter the TNB frame briefly when they study P-chem and the motion of subatomic particles.

 
\section{Curvature and Torsion}

We already know that $\vec T=\dfrac{d\vec r}{ds}$ has length 1. This means that if we move along the curve $\vec r$ using $s$ as our parameter (not $t$), then we move along the curve at a constant speed of 1. The fact that we are moving at speed 1 means that we can study the properties of the curve without having to worry about our speed. We would like to know how sharp a corner is (which we'll call the curvature). To determine how sharp a corner is, we must forget about speed for a bit. If we encounter a really tight corner (so a rapid change in direction over a very short distance) we would expect $\dfrac{d\vec T}{ds}$ to be a fairly long vector. A small change in $s$ results in a large change in $T$. However, if we were to move along this tight corner at a really slow speed, we would expect $\frac{d\vec T}{dt}$ to be a really small vector. A small change in $t$ would not produce much change in $T$. 

\begin{problem}\marginpar{Watch a \href{http://www.youtube.com/watch?v=aJttU3kS_p8&list=PL30EE81142B1ED1F0&index=7&feature=plpp_video}{YouTube Video}.  }%
Suppose we are traveling along the space curve $\vec r$, and we know the unit tangent vector is $\vec T$. 
\begin{enumerate}
 \item If we are moving along a straight line, then what is $\dfrac{d\vec T}{ds}$? Explain.
 \item If we veer slightly off a straight line, should $\dfrac{d\vec T}{ds}$ be large or small? Why?
 \item If we veer slightly off a straight line, and are moving extremely slow, should $\dfrac{d\vec T}{dt}$ be large or small? Explain.
 \item If we veer slightly off a straight line, and are moving extremely fast, should $\dfrac{d\vec T}{dt}$ be large or small? Explain.
 \item If we know $\dfrac{d\vec T}{ds}$ has length $\frac{1}{2}$, and our speed is $50$, how long is $\dfrac{d\vec T}{dt}$? Explain. [Hint: remember that $\dfrac{d\vec T}{ds} = \dfrac{d\vec T/dt}{ds/dt}$, and we've seen $ds/dt$ before.]
\end{enumerate}
\end{problem}

We will often be computing derivatives with respect to $s$, instead of $t$, because we want to determine physical properties about the curve. Moving really slowly around a tight corner won't produce a large tagent vector because our speed is slow.  Similarly, moving quickly along a curve that hardly changes could produce a misleading large tangent vector.  However, if we remove the speed from the problem, by taking a derivative with respect to $s$ instead of $t$, then we'll learn how quickly the curve veers away from $\vec T$ as we increase in length.  
When we compute $\dfrac{d\vec N}{ds}$, we will find how rapidly $\vec N$ rotates away from the plane containing $T$ and $N$ (motion and acceleration). 
When we compute $\dfrac{d\vec B}{ds}$, we will find how rapidly $\vec B$ rotates.  We'll show that both $\dfrac{d\vec N}{ds}$ and $\dfrac{d\vec B}{ds}$ cause a rotation of $\vec N$ and $\vec B$ about the tangent vector $\vec T$. The magnitude of this rotation, as $\vec B$ wraps around $\vec T$ counterclockwise, is called the torsion. Let's formally define curvature and torsion.

\begin{definition}[Curvature and Torsion]
 Let $\vec r(t)$ be a parametrization of a smooth curve $C$ with unit tangent vector $\vec T(t)$.  The curvature vector, written $\vec \kappa(t)$, is the derivative of $\vec T$ with respect to arc length, which means 
 $$\vec \kappa(t)=\dfrac{d\vec T}{ds}=\dfrac{d\vec T/dt}{ds/dt}=\dfrac{d\vec T/dt}{|d\vec r/dt|}.$$ 
 The length of the curvature vector is the curvature, written $\kappa = |\vec\kappa|$. Notice that $\kappa$ is a number.

 The derivative of $\vec B$ with respect to $s$ tells us how rapidly the plane containing $\vec T$ and $\vec N$ rotates. We'll define the torsion vector to be 
\marginpar{Watch a \href{http://www.youtube.com/watch?v=MVtUc2peJn0&feature=bf_next&list=PL30EE81142B1ED1F0&lf=plpp_video}{YouTube Video}.  }%
 $$\vec \tau = \dfrac{d\vec B}{ds} = \dfrac{d\vec B/dt}{ds/dt}=\dfrac{d\vec B/dt}{|d\vec r/dt|}.$$ 
 The torsion $\tau$, up to a sign, is the length of this vector. We say there is positive torsion if $\vec \tau$ causes a counterclockwise rotation about $\vec T$, which occurs precisely when $\vec tau$ and $\vec N$ point in opposite directions. We can summarize this is $$\tau=\left|\dfrac{d\vec B}{ds}\right|\quad \text{or}\quad \tau=-\left|\dfrac{d\vec B}{ds}\right|,$$ where you choose ``$+$'' if $\vec N$ and $\vec \tau$ point in opposite directions. 
\end{definition}


\begin{problem}
\marginparbmw{See 13.4: 9-16 and 13.5: 9-16 (the relevant parts) for more practice.}%
Consider the helix $r(t)=(3\cos t, 3\sin t, 4t)$. In problem \ref{helix example of T N and B} we found $\vec T$, $\vec N$, and $\vec B$. Compute both $\vec \kappa=\dfrac{d\vec T}{ds}$ and $\vec \tau=\dfrac{d\vec B}{ds}$, and then give $\kappa$ and $\tau$.
\end{problem}

\begin{problem}
 Consider the helix $r(t)=(4\sin t, 4\cos t, 3t)$. Use a computer to find $\vec T$, $\vec N$, $\vec B$, $\vec \kappa$, and $\vec \tau$. Use your answers to then give $\kappa$ and $\tau$. (When you present on the board, just write down the 5 vectors, and then explain how you obtained $\kappa$ and $\tau$ from these vectors.)
\end{problem}

In both examples above, you should have noticed that $\vec \tau$ was either parallel to $\vec N$ or anti-parallel to $\vec N$.  We'll now show this is always the case.

\begin{problem}\marginpar{Watch a \href{http://www.youtube.com/watch?v=MVtUc2peJn0&feature=bf_next&list=PL30EE81142B1ED1F0&lf=plpp_video}{YouTube Video}.  }%
 Suppose a curve $\vec r(t)$ has the frame $\vec T(t)$, $\vec N(t)$, and $\vec B(t)$. Prove that $\dfrac{d\vec B}{ds}$ is either parallel to $\vec N$, or points opposite $\vec N$. Here are some steps.
 \begin{itemize}
  \item Why is $\dfrac{d\vec B}{ds}$ orthogonal to $\vec B$? [Hint: How long is $\vec B$? Use a key theorem from earlier.]
  \item We know $\vec B=\vec T\times \vec N$. Compute the derivative of both sides using the product rule. Explain why $\frac{d\vec T}{ds}\times \vec N$ cancels out. Then explain why $\dfrac{d\vec B}{ds}$ is orthogonal to $\vec T$.
  \item If $\dfrac{d\vec B}{ds}$ is orthogonal to both $\vec B$ and $\vec T$ why must it be either parallel or anti-parallel to $\vec N$?
 \end{itemize}
\end{problem}


When the curvature is nonzero, the curve bends away from the direction of motion.  We could use a circle to approximate how great this bend is. A small change in direction would require a large circle.  A large change in direction would require a small circle. 
What we want is to find a circle that best approximates the curve (kind of like a Taylor polynomial, only now we'll use a circle.) We want the circle to meet the curve $\vec r$ tangentially, and we want the curvature of the circle to match the curvature of the curve.  The next problem shows you the relationship between the radius $\rho$ of this circle and the curvature $\kappa$ of the curve.

\begin{problem}
 Consider the curve $\vec r(t)=(a\cos t, a\sin t)$.
 \begin{enumerate}
  \item Draw the curve, and state the radius $\rho$ of the best approximating circle.
  \item Find the curvature $\vec \kappa$ by performing a computation.
  \item What relationship exists between $\rho$ and $\kappa$?  If the radius $\rho$ were to increase, what would happen to $\kappa$?
 \end{enumerate}
\end{problem}


\begin{definition}[Circle and Center of Curvature]\marginpar{Watch a \href{http://www.youtube.com/watch?v=cHez5K1EWPs&list=PL30EE81142B1ED1F0&index=8&feature=plpp_video}{YouTube Video}.  }%
When the curvature $\kappa$ of a smooth curve is nonzero, we'll define the radius of curvature, written $\rho$, to be the reciprocal $\rho = \dfrac{1}{\kappa}$. The center of curvature is the center of this circle.
\end{definition}

\begin{problem}
 Consider the curve $\vec r(t)=(t,\sin 3 t)$. Find the radius and center of curvature at $t=\pi/6$ (*see the suggestion below). Draw the curve, and draw the circle of curvature at $t=\pi/6$. 
 %Without doing any more computations, what are the radius and center of curvature at $t=\pi/2$?  How about at $t=\pi/3$? 
 (You will have shown why the center of curvature is at $\vec r + \rho \vec N$.)
 
*The computations here can get pretty ugly. After getting the unit tangent vector $\ds \vec T(t) = \frac{(1,3\cos 3t)}{\sqrt{1+9\cos^2(3t)}}$, you will need to compute $d\vec T/dt$. Just use the quotient rule (don't try to simplify, rather just write out the big mess that comes from the quotient rule).  Then immediately plug in $t=\pi/6$ into $d\vec T/dt$ before trying to find $\vec \kappa$ and $\rho$ at $t=\pi/6$. Most of complication will disappear. Another option is to use problem   \ref{formula for curvature}. 
\end{problem}

\begin{problem}
 Consider the helix $\vec r(t)=(t,\sin t,\cos t)$. Find the radius of curvature $t=\pi/2$. Draw the curve, and draw the circle of curvature at $t=\pi/2$. Then find the center of curvature at $t=\pi/2$. Guess the center of curvature at $t=\pi$?
\end{problem}

Here's two final problem related to curvature.  They provide a really easy way to compute the curvature of a function of the form $y=f(x)$, and of any curve in the plane. Coming up with the formulas is not necessarily easy, but using them is fairly quick. This formula gets used in dynamics, and shows up on the Fundamentals of Engineering exam (where you just have to use the formula, not prove where it comes from).
\begin{problem}\label{formula for curvature}
\marginparbmw{See 13.4: 5.}%
 The function $y=f(x)$ can be given the parametrization  $\vec r(x) = (x,f(x))$.  Use this parametrization to show that the curvature is $$\kappa(x) = \frac{|f''(x)|}{(1+(f')^2)^{3/2}}.$$
\end{problem}

%\begin{problem*}[Optional]
%\marginparbmw{See 13.4: 6. Though this problem is long, it is perhaps one of the best problems to do.}%
% Suppose a smooth curve has the parametrization $\vec r(t) = (x(t),y(t))$.  Use this parametrization to show that the curvature is %$$\kappa(t) = \frac{|x'y''-y'x''|}{((x')^2+(y')^2)^{3/2}} = \frac{|\dot{x} \ddot{y}-\dot{y} \ddot{x}|}{(\dot{x}^2+\dot {y}^2)^{3/2}} .$$ The dot notation is just an alternate way of saying, ``Take the derivative with respect to time.'' If $x=t$, then note that this gives the previous formula.
%\end{problem*}


When a civil engineering team builds a road, they have to pay attention to the curvature of the road.  If the curvature of the road is too large, accidents will happen and the civil engineering team will be liable. How do they make sure the curvature never gets to large?  They use the circle of curvature. When they want to cause a road to turn, they'll find the center of curvature, send a surveyor out to the center, and then have the surveyor make sure that the road follows the circle of curvature for a short distance. They actually pace out the circle of curvature and then build the road along this circle for a hundred feet or so.  Then, they recompute the radius of curvature (if they need the direction to change again), and pace out another circle.  In this way, they can guarantee that the curvature never gets large. In the next section we'll see how curvature is directly related to normal acceleration (which is what causes semis to tip, and vehicles to slide off icy roads.)

\section{Tangential and Normal Components of Acceleration}

In this section, we'll show that you write the acceleration of an object moving along a curve $\vec r(t)$ with velocity $\vec v(t)$ as the sum
$$\vec a(t) = a_T\vec T+a_N\vec N=\frac{d}{dt}|\vec v(t)| \vec T + \kappa |\vec v|^2 \vec N.$$
The scalars $a_T=\dfrac{d}{dt}|\vec v(t)|$ and $a_N=\kappa |\vec v|^2$ 
\marginpar{Engineers often use the equivalent formula $a_N = \frac{|\vec v|^2}{\rho}$, as $\rho$ is a physical distance that they can measure.} 
are called the tangential and normal components of acceleration.  All we are doing is writing the vector $\vec a(t)$ as the sum of a vector parallel to $\vec T$ and a vector orthogonal to $\vec T$. Before we decompose the acceleration into its tangential and normal components, let's look at two examples to see what these facts physically represent.

\begin{problem}
\marginparbmw{See 13.5: 17-20 for more practice.}%
 Consider the path of an object in projectile motion that has been fired from the origin. Draw a typical path followed by a projectile.  The acceleration $\vec a(t)=(0,-g)$ acts straight down for any time $t$.  
\begin{itemize}
 \item Pick a point on your path before the max height occurs. At that point, draw both $\vec T$, $\vec a$, and the projection of $\vec a$ onto $\vec T$.  Is $a_T$ positive or negative? 
 \item At the point you chose above, is the speed of the projectile increasing or decreasing as it climbs higher? Why is it reasonable to believe $a_T = \frac{d}{dt}|\vec v(t)|$? Explain.
 \item Now pick a point after the projectile passes the peak.  Then repeat the last two parts at this point.
\end{itemize}
\end{problem}

\begin{problem}
 Imagine that you are riding as a passenger on a road and encounter a series of switchbacks (so the road starts to zigzag up the mountain). Right before each bend in the road, you see a yellow sign that tells you a U-turn is coming up, and that you should reduce your speed from 45 mi/hr to 15 mi/hr.  Assume the largest curvature along the turn is $\kappa$. Recall that $a_N=\kappa |\vec v|^2$. The engineers of the road designed the road so that if you are moving at 15 mi/hr, then the normal acceleration will be at most $A$ units. 
\begin{enumerate}
 \item Suppose that your driver (Ben) ignores the suggestion to slow down to 15 mi/hr.  He keeps going 45 mi/hr through the turn. Had he slowed down, the max acceleration would be $A$.  You're traveling 3 times faster than suggested.  What will your maximum normal acceleration be? [It's more than $3A$.]
 \item You yell at Ben to slow down (you don't want to die). So Ben decides to only slow to 30 mi/hr. He figures this means you'll only feel twice as much acceleration as $A$.  Explain why this line of reasoning is flawed.
 \item Ben gets frustrated by the fact that he has to slow down. He complains about the engineers who designed the road, and says, ``they should have just built a larger corner so I could keep going 45.''  How much larger should the radius of the circle be so that you can travel 45 mi/hr instead of 15 mi/hr, and still feel the same acceleration $A$?
 \item Which will cause the normal acceleration to decrease more, halving your speed or halving the curvature (doubling the radius)?
\end{enumerate}
\end{problem}

% \begin{problem}
%  We defined the principle unit normal vector as $\vec N = \dfrac{d\vec T/dt}{|d\vec T/dt|}$.  Explain why we can write $\vec N = \dfrac{d\vec T/ds}{|d\vec T/ds|}$ as well. Then use this fact to explain why , which means we can write $\ds\kappa|\vec v|\vec N=\frac{d\vec T}{dt}$.  
% \end{problem}


\begin{problem}\marginpar{Watch a \href{http://www.youtube.com/watch?v=cSh2Bdd-yTg&feature=bf_next&list=PL30EE81142B1ED1F0&lf=plpp_video}{YouTube Video}.  }%
 Prove that $\ds \vec a(t) = a_T\vec T+a_N\vec N=\frac{d}{dt}|\vec v| \vec T + \kappa |\vec v|^2 \vec N.$ Here's some hints.
\begin{itemize}
 \item Rewrite the velocity $\vec v$ as a magnitude $|\vec v|$ times a direction $\vec T$.  
 \item We know that $\vec a(t) = \frac{d}{dt}\vec v(t)$ (acceleration is the derivative of velocity). Take the derivative of $\vec v = |\vec v|\vec T$ by using the product rule (on the scalar product $|\vec v|\vec T$).
 \item You should encounter the quantity $d\vec T/dt$ somewhere in your product. Write this quantity as a magnitude times a direction. [We've seen $d\vec T/dt$ in much of our previous work. You'll need to prove that $\ds d\vec T/dt=\kappa|\vec v|\vec N$.]
\end{itemize}
\end{problem}

\begin{problem}
We now know that $\ds \vec a = a_T\vec T+a_N\vec N=\frac{d}{dt}|\vec v| \vec T + \kappa |\vec v|^2 \vec N.$  
Use this to prove that the curvature can be obtained from the formula $$\kappa = \frac{|\vec v\times \vec a|}{|\vec v|^3} = \frac{|\vec r'\times \vec r''|}{|\vec r'|^3}.$$ [Hint: cross both sides with $\vec v$, simplify, take the magnitude of each side, and solve for $\kappa$.]
\end{problem}



